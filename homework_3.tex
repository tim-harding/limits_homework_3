\documentclass[12pt]{article}

\usepackage{fouriernc}
\usepackage{amssymb}
\usepackage{amsmath}
\usepackage{amsfonts}
\usepackage[utf8]{inputenc}
\usepackage[T1]{fontenc}
\usepackage[margin=1in]{geometry}
\usepackage{graphicx}

\graphicspath{ {./images/} }

\newcommand{\curly}[1]{\left\{      #1 \right\}     }
\newcommand{\round}[1]{\left(       #1 \right)      }
\newcommand{\hard} [1]{\left[       #1 \right]      }
\newcommand{\abs}  [1]{\left|       #1 \right|      }
\newcommand{\floor}[1]{\left\lfloor #1 \right\rfloor}
\newcommand{\ceil} [1]{\left\lceil  #1 \right\rceil }
\newcommand{\R}    [0]{\mathbb{R}                   }
\newcommand{\Z}    [0]{\mathbb{Z}                   }

\setlength{\parindent}{0in}

\title{Homework 3}
\author{Tim Harding}

\begin{document}
\maketitle

\section*{2.4.8}

\subsection*{Problem}

Prove that $\lim_{x \to x_0} x = x_0$.

\subsection*{Solution}

$\forall \epsilon > 0,\ \exists \delta = \epsilon,\ \forall x : 0 < \abs{x - x_0} < \delta,\ \abs{x - x_0} < \epsilon$



\section*{2.4.9}

\subsection*{Problem}

Prove that if $x_0, k \in \R$ then $\lim_{x \to x_0} k = k$.

\subsection*{Solution}

$\forall \epsilon > 0,\ \exists \delta = \epsilon,\ \forall x : 0 < \abs{x - x_0} < \delta,\ \abs{k - k} = 0 < \epsilon$



\section*{2.4.10}

\subsection*{Problem}

Prove the limit for $\lim_{x \to 2} \round{x^3 - 2x^2 + x - 7}$.

\subsection*{Solution}

The limit is $2^3 - 2 \times 2^2 + 2 - 7 = -5$.

$\forall \epsilon > 0,\ \exists \delta = \min\round{1, \frac{\epsilon}{10}},\ \forall x : 0 < \abs{x - 2} < \delta,\ \\ \abs{x^3 - 2x^2 + x - 7 - (-5)} = \abs{x - 2} \times \abs{x^2 + 1} < 10 \abs{x - 2} < \epsilon$

This can also be found using the sum and product laws.



\section*{2.4.14}

\subsection*{Problem}

Show that given $\lim_{x \to x_0} f(x) = L$ and $\lim_{x \to x_0} f(x) = l$ we have $L = l$.

\subsection*{Solution}

$\forall \epsilon > 0,\ \exists \delta = \_\_\_\_,\ \forall x : 0 < \abs{x - x_0} < \delta,\ \abs{f(x) - L} < \epsilon,\ \abs{f(x) - l} < \epsilon,\ \\ L - l = \abs{L - l} = \abs{-f(x) + L + f(x) - l} = \abs{(f(x) - l) + -(f(x) - L)} \leq \\ \abs{f(x) - l)} + \abs{-(f(x) - L)} = \abs{f(x) - l} + \abs{f(x) - L} < 2\epsilon$. By choosing $\epsilon$ arbitrarily small, we find that the difference between $L$ and $l$ is arbitrarily small. Thus, $L$ and $l$ are arbitrarily close together, so $L = l$.



\section*{2.4.15}

\subsection*{Problem}

Prove that $\lim_{x \to x_0} f(x) = \infty \implies \nexists L \in \R : \lim_{x \to x_0} f(x) = L$.

\subsection*{Solution}

$\forall M > 0,\ \exists \delta,\ \forall x < \abs{x - x_0} < \delta,\ f(x) > M$. Suppose for the sake of contradiction that $\exists L : \forall \epsilon > 0,\ \exists \bar{\delta},\ \forall x : x < \abs{x - x_0} < \bar{\delta},\ \abs{f(x) - L} < \epsilon$. If we choose that $\epsilon = 1$ and $M = L + 2$, then $f(x) > L + 2$ and $\abs{f(x) - L} < 1$. However, this would imply that $f(x)$ and $L$ are simultaneously greater than 2 apart and less than 1 apart, a contradiction.



\section*{2.4.16}

\subsection*{Problem}

\subsection*{Solution}






\end{document}

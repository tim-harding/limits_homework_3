\documentclass[12pt]{article}

\usepackage{fouriernc}
\usepackage{amssymb}
\usepackage{amsmath}
\usepackage{amsfonts}
\usepackage[utf8]{inputenc}
\usepackage[T1]{fontenc}
\usepackage[margin=1in]{geometry}
\usepackage{graphicx}

\graphicspath{ {./images/} }

\newcommand{\curly}[1]{\left\{      #1 \right\}     }
\newcommand{\round}[1]{\left(       #1 \right)      }
\newcommand{\hard} [1]{\left[       #1 \right]      }
\newcommand{\abs}  [1]{\left|       #1 \right|      }
\newcommand{\floor}[1]{\left\lfloor #1 \right\rfloor}
\newcommand{\ceil} [1]{\left\lceil  #1 \right\rceil }
\newcommand{\R}    [0]{\mathbb{R}                   }
\newcommand{\Z}    [0]{\mathbb{Z}                   }
\newcommand{\N}    [0]{\mathbb{N}                   }

\setlength{\parindent}{0in}

\title{Homework 3}
\author{Tim Harding}

\begin{document}
\maketitle

\section*{2.4.8}

\subsection*{Problem}
Prove that $\lim_{x \to x_0} x = x_0$.

\subsection*{Solution}
$\forall \epsilon > 0,\ \exists \delta = \epsilon,\ \forall x : 0 < \abs{x - x_0} < \delta,\ \abs{x - x_0} < \epsilon$



\section*{2.4.9}

\subsection*{Problem}
Prove that if $x_0, k \in \R$ then $\lim_{x \to x_0} k = k$.

\subsection*{Solution}
$\forall \epsilon > 0,\ \exists \delta = \epsilon,\ \forall x : 0 < \abs{x - x_0} < \delta,\ \abs{k - k} = 0 < \epsilon$



\section*{2.4.10}

\subsection*{Problem}
Prove the limit for $\lim_{x \to 2} \round{x^3 - 2x^2 + x - 7}$.

\subsection*{Solution}
The limit is $2^3 - 2 \times 2^2 + 2 - 7 = -5$.

$\forall \epsilon > 0,\ \exists \delta = \min\round{1, \frac{\epsilon}{10}},\ \forall x : 0 < \abs{x - 2} < \delta,\ \\ \abs{x^3 - 2x^2 + x - 7 - (-5)} = \abs{x - 2} \times \abs{x^2 + 1} < 10 \abs{x - 2} < \epsilon$

This can also be found using the sum and product laws.



\section*{2.4.14}

\subsection*{Problem}
Show that given $\lim_{x \to x_0} f(x) = L$ and $\lim_{x \to x_0} f(x) = l$ we have $L = l$.

\subsection*{Solution}
$\forall \epsilon > 0,\ \exists \delta = \_\_\_\_,\ \forall x : 0 < \abs{x - x_0} < \delta,\ \abs{f(x) - L} < \epsilon,\ \abs{f(x) - l} < \epsilon,\ \\ L - l = \abs{L - l} = \abs{-f(x) + L + f(x) - l} = \abs{(f(x) - l) + -(f(x) - L)} \leq \\ \abs{f(x) - l)} + \abs{-(f(x) - L)} = \abs{f(x) - l} + \abs{f(x) - L} < 2\epsilon$. By choosing $\epsilon$ arbitrarily small, we find that the difference between $L$ and $l$ is arbitrarily small. Thus, $L$ and $l$ are arbitrarily close together, so $L = l$.



\section*{2.4.15}

\subsection*{Problem}
Prove that $\lim_{x \to x_0} f(x) = \infty \implies \nexists L \in \R : \lim_{x \to x_0} f(x) = L$.

\subsection*{Solution}
$\forall M > 0,\ \exists \delta,\ \forall x < \abs{x - x_0} < \delta,\ f(x) > M$. Suppose for the sake of contradiction that $\exists L : \forall \epsilon > 0,\ \exists \bar{\delta},\ \forall x : x < \abs{x - x_0} < \bar{\delta},\ \abs{f(x) - L} < \epsilon$. If we choose that $\epsilon = 1$ and $M = L + 2$, then $f(x) > L + 2$ and $\abs{f(x) - L} < 1$. However, this would imply that $f(x)$ and $L$ are simultaneously greater than 2 apart and less than 1 apart, a contradiction.



\section*{2.4.16}

\subsection*{Problem}
Prove the Squeeze Theorem.

\subsection*{Solution}
Given some interval $I$, suppose that $\forall x \in I$, $x_0 \in I$, $f(x) \leq g(x) \leq h(x)$, $\lim_{x \to x_0} f(x) = L$, and $\lim_{x \to x_0} h(x) = L$. $\forall \epsilon > 0$, $\exists \delta_1, \delta_2$, $\forall x : 0 < \abs{x - x_0} < \delta_1 \implies \abs{f(x) - L} < \epsilon$, $\forall x : 0 < \abs{x - x_0} < \delta_2 \implies \abs{h(x) - L} < \epsilon$. Note that $-\epsilon < f(x) - L < \epsilon$ and $-\epsilon < h(x) - L < \epsilon$. In particular,
\begin{align*}
    -\epsilon &< f(x) - L \\
    L - \epsilon &< f(x)
\end{align*}
and
\begin{align*}
    h(x) - L &< \epsilon \\
    h(x) &< L + \epsilon
\end{align*}
With $\delta = \min(\delta_1, \delta_2)$, we have $\forall x : 0 < \abs{x - x_0} < \delta \implies L - \epsilon < f(x) < g(x) < h(x) < L + \epsilon$, so $-\epsilon < g(x) - L < \epsilon$ which means that $\abs{g(x) - L} < \epsilon$ and $\lim_{x \to x_0} g(x) = L$.



\section*{2.4.17}

\subsection*{Problem}
Prove that $\lim_{x \to x_0} f(x) = L$, $L \geq 0 \implies \lim_{x \to x_0} \sqrt{f(x)} = \sqrt{L}$.

\subsection*{Solution}
$\forall \epsilon > 0$, $\exists \delta$, $\forall x : 0 < \abs{x - x_0} < \delta$, $\abs{f(x) - L} < \epsilon$.

\begin{align*}
    \abs{\sqrt{f(x)} - \sqrt{L}} \\
    \abs{\round{\sqrt{f(x)} - \sqrt{L}} \frac{\sqrt{f(x)} + \sqrt{L}}{\sqrt{f(x)} + \sqrt{L}}} \\
    \abs{\frac{f(x) - L}{\sqrt{f(x)} + \sqrt{L}}} \\
    \abs{f(x) - L} \abs{\frac{1}{\sqrt{f(x)} + \sqrt{L}}}
\end{align*}



\section*{2.4.B}

\subsection*{Problem}
State the $\frac{\infty}{0}$ type of quotient law for $x \to x_0^+$.

\subsection*{Solution}

Since 0 in the denominator pushes the $\infty$ in the numerator further from zero, we know the quotient will be arbitrarily large. Since we approach the limit from the right, the quotient is positive and $\frac{\infty}{0} = \infty$.



\section*{2.5.A}

\subsection*{Problem}
\begin{align*}
    \lim_{x \to 4} \frac{x-4}{\sqrt{x} - 2}
\end{align*}

\subsection*{Solution}
The form of the limit is
\begin{align*}
    \frac{4 - 4}{\sqrt{4} - 2} = \frac{0}{0}
\end{align*}
We apply L'Hopital's rule:
\begin{align*}
       &   \lim_{x \to 4} \frac{1}{\frac{1}{2} \frac{1}{\sqrt{x}}} \\
    =\ & 2 \lim_{x \to 4} \sqrt{x} \\
    =\ & 2 \times \sqrt{4} \\
    =\ & 4
\end{align*}



\section*{2.5.B}

\subsection*{Problem}
\begin{align*}
    \lim_{x \to 0} \frac{\cos(x) - 1}{x^2}
\end{align*}

\subsection*{Solution}
The form of the limit is
\begin{align*}
    \frac{\cos(0) - 1}{0^2} = \frac{0}{0}
\end{align*}
We apply L'Hopital's rule:
\begin{align*}
    & \lim_{x \to 0} \frac{-\sin{x}}{2x} \\
    =& \lim_{x \to 0} \frac{-\cos(x)}{2} \\
    =& -\frac{1}{2} \lim_{x \to 0} \cos(x) \\
    =& -\frac{1}{2} \cos(0) \\
    =& -\frac{1}{2}
\end{align*}




\section*{2.5.C}

\subsection*{Problem}
\begin{align*}
    \lim_{x \to 0^+} x \ln^2(x)
\end{align*}

\subsection*{Solution}
We rewrite as a quotient:
\begin{align*}
    \lim_{x \to 0^+} \frac{(\ln(x))^2}{\frac{1}{x}}
\end{align*}
This limit is of the form
\begin{align*}
    \frac{(\ln(0))^2}{\frac{1}{0}} = \frac{(-\infty)^2}{\infty} = \frac{\infty}{\infty}
\end{align*}
We apply L'Hopital's rule:
\begin{align*}
     & \lim_{x \to 0^+} \frac{2 \ln(x) \frac{1}{x}}{-\frac{1}{x^2}} \\
    =& -2 \lim_{x \to 0^+} x \ln(x)
\end{align*}
We have a limit of the form $0 \times -\infty$, so we continue:
\begin{align*}
       & -2 \lim_{x \to 0^+} \frac{\ln(x)}{\frac{1}{x}} \\
    =\ & -2 \lim_{x \to 0^+} \frac{\frac{1}{x}}{-\frac{1}{x^2}} \\
    =\ & 2 \lim_{x \to 0^+} x \\
    =\ & 2 \times 0 \\
    =\ & 0
\end{align*}



\section*{2.5.D}

\subsection*{Problem}
\begin{align*}
    \lim_{x \to 0^+} x^{\sin(x)}
\end{align*}

\subsection*{Solution}




\section*{2.5.E}

\subsection*{Problem}
\begin{align*}
    \lim_{x \to 0^+} x^{x^x}
\end{align*}

\subsection*{Solution}




\section*{3.1.6}

\subsection*{Problem}
Show the first 6 terms of $b_n = \frac{1}{n}$ and show two graphical representations.

\subsection*{Solution}
$1$, $\frac{1}{2}$, $\frac{1}{3}$, $\frac{1}{4}$, $\frac{1}{5}$, $\frac{1}{6}$.



\section*{3.1.8}

\subsection*{Problem}
Find a pattern in the sequence $-1$, $\frac{1}{4}$, $-\frac{1}{9}$, $\frac{1}{16}$, $-\frac{1}{25}$.

\subsection*{Solution}
$c_n = (-1)^n \times n^{-2}$. This is similar to $b_n$ in that it approaches 0 but different in that it does so more quickly while leaping back and forth over the asymptote at each step.



\section*{3.1.9}

\subsection*{Problem}
Find an explicit formula for the recursive relation
\begin{align*}
    a_n = \begin{cases}
        a_1 &= 0 \\
        a_{n + 1} &= a_n - 1
    \end{cases}
\end{align*}

\subsection*{Solution}
\begin{align*}
    \sum_{i = 1}^n a_i &= 0 + (a_1 - 1) + (a_2 - 1) + \cdots + (a_{n - 1} - 1) \\
    a_n &= -1 \times (n - 1) \\
    a_n &= 1 - n
\end{align*}



\section*{3.1.10}

\subsection*{Problem}
Find an explicit formula for the recursive relation
\begin{align*}
    a_n = \begin{cases}
        a_1 &= 1 \\
        a_{n + 1} &= a_n + n + 1
    \end{cases}
\end{align*}

\subsection*{Solution}
\begin{align*}
    \sum_{i = 1}^n a_i &= 1 + (a_1 + 1 + 1) + (a_2 + 2 + 1) + (a_3 + 3 + 1) + \cdots + (a_{n - 1} + n - 1 + 1) \\
    a_n &= \sum_{i = 1}^n 1 + \sum_{i = 1}^{n - 1} i \\
    a_n &= n + \frac{(1 + (n - 1)) (n - 1)}{2} \\
    a_n &= n + \frac{n(n - 1)}{2}
\end{align*}



\section*{3.2.8}

\subsection*{Problem}
Prove that the sequence $\frac{1}{2}$, $-\frac{1}{5}$, $\frac{1}{8}$, $-\frac{1}{11}$, $\frac{1}{14}$ converges to zero.

\subsection*{Solution}
We observe that the pattern, starting with $n = 0$, is
\begin{align*}
    (-1)^n \frac{1}{2 + 3n}
\end{align*}
We then proceeed with the the proof that $\lim_{n \to \infty} (-1)^n \frac{1}{2 + 3n} = 0$.

$\forall \epsilon \in \R : \epsilon > 0$, $\exists N = \ceil{\frac{1}{3 \epsilon}}$, $\forall n \in \N : n > N$, $\abs{(-1)^n \frac{1}{2 + 3n} - 0} < \abs{\frac{(-1)^n}{3n}} = \frac{1}{3n} < \epsilon$



\section*{3.2.9}

\subsection*{Problem}
If the sequence $\frac{\cos n}{n}$ converges, prove its limit.

\subsection*{Solution}
$\forall \epsilon \in \R : \epsilon > 0$, $\exists N = \frac{1}{\epsilon}$, $\forall n \in \N : n > N$, $\abs{\frac{\cos n}{n} - 0} \leq \frac{1}{n} < \epsilon$



\section*{3.2.10}

\subsection*{Problem}
Investigate the convergence of $\floor{1 - \frac{1}{n}}$.

\subsection*{Solution}



\section*{3.2.11}

\subsection*{Problem}

\subsection*{Solution}



\section*{3.2.12}

\subsection*{Problem}

\subsection*{Solution}



\section*{3.2.15}

\subsection*{Problem}

\subsection*{Solution}



\section*{3.2.17}

\subsection*{Problem}

\subsection*{Solution}



\end{document}
